%  \documentclass[10pt]{article}
% %\usepackage[margin=.4in]{geometry} 
% \usepackage[left=.25in,top=.25in,right=.25in,head=.3in,foot=.3in]{geometry}
% \usepackage[pdftex]{graphicx} 
% \usepackage{epstopdf}  
% \usepackage{verbatim}
% \usepackage{amssymb} 
% \usepackage{setspace}   
% \usepackage{longtale}  

% \newenvironment{cin}[1]{
% \begin{center}
%  \input{#1}  

% \end{center}}


% \usepackage{natbib}
% \bibpunct{(}{)}{,}{a}{,}{,}


%on *part1_set_up
\documentclass[11pt]{article}      
\usepackage[margin=10pt,font=small,labelfont=bf]{caption}[2007/03/09]

%\long\def\symbolfootnote[#1]#2{\begingroup% 							%these can be used to make footnote  nonnumeric asterick, dagger etc
%\def\thefootnote{\fnsymbol{footnote}}\footnote[#1]{#2}\endgroup}		%see: http://help-csli.stanford.edu/tex/latex-footnotes.shtml
%%
%#  \abovecaptionskip: space above caption
%# \belowcaptionskip: space below caption
%%
% %
\usepackage{setspace}
\usepackage{longtable}
%\usepackage{natbib}
\usepackage{anysize}
% %
\usepackage{natbib}
\bibpunct{(}{)}{,}{a}{,}{,}

\usepackage{amsmath} % Typical maths resource packages
\usepackage[pdftex]{graphicx}                 % Packages to allow inclusion of graphics
%\usepackage{color}					% For creating coloured text and background
\usepackage{epstopdf}
\usepackage{hyperref}                 % For creating hyperlinks in cross references
\usepackage{color}


% \hypersetup{
%     colorlinks = true,
%     linkcolor = red,
%     anchorcolor = red,
%     citecolor = blue,
%     filecolor = red,
%     pagecolor = red,
%     urlcolor = red
% }


% \newenvironment{rr}[1]{
% \hspace{-1in}
% \texttt{[{#1}]} 
% \hspace{.1in}}
% \newenvironment{rr}[1]{
% % \hspace{-.475in}
% % $>>>$
% }

\usepackage{changepage}   % for the adjustwidth environment
\newenvironment{cc}[1]{ 
\begin{adjustwidth}{-1.5cm}{}
  \vspace{.3 in}
  {\color{blue} \footnotesize  {#1}}
  \vspace{.1in}
\end{adjustwidth}
}






%\oddsidemargin 0cm
%\evensidemargin 0cm
%\pagestyle{myheadings}         % Option to put page headers
                               % Needed \documentclass[a4paper,twoside]{article}
%\markboth{{\small\it Politics and Life Satisfaction }}
%{{\small\it Adam Okulicz-Kozaryn} }
\marginsize{2cm}{2cm}{0cm}{1cm} %\marginsize{left}{right}{top}{bottom}:
\renewcommand\familydefault{\sfdefault}
%\headsep 1.5cm

% \pagestyle{empty}			% no page numbers
% \parindent  15.mm			% indent paragraph by this much
% \parskip     2.mm			% space between paragraphs
% \mathindent 20.mm			% indent matheistequations by this much

					% Helps LaTeX put figures where YOU want
 \renewcommand{\topfraction}{.9}	% 90% of page top can be a float
 \renewcommand{\bottomfraction}{.9}	% 90% of page bottom can be a float
 \renewcommand{\textfraction}{0.1}	% only 10% of page must to be text

% no section number display
%\makeatletter
%\def\@seccntformat#1{}
%\makeatother
% no numbers in toc
%\renewcommand{\numberline}[1]{}
 
\newenvironment{ig}[1]{
\begin{center}
 %\includegraphics[height=5.0in]{#1} 
 \includegraphics[height=3.3in]{#1} 
\end{center}}

\usepackage{pdfpages}

%this disables indenting--looks cleraner that way
\newlength\tindent
\setlength{\tindent}{\parindent}
\setlength{\parindent}{0pt}
\renewcommand{\indent}{\hspace*{\tindent}}


\date{{}\today}
\title{Author's response\\ {\large Manuscript Number: JOHS-D-17-00078 \\ Title:  Energy Use And Happiness }}

\author{}
%off
%on *part2_intro
\begin{document}
\bibliographystyle{/home/aok/papers/root/tex/ecta}
\maketitle

\tableofcontents

\section{Response to Editor} 


%TODO be more conversational here ! :)
\noindent Dear Professor Rossouw,\\

\noindent Thank you for the opportunity to submit a revised draft.
We list below in inline format my brief responses to reviewers'
 comments and attach at the end tracked changes that
 show precisely the additions and deletions.\\
 
 We note that the  heart of this paper is exploration of a core bivariate relationship between
 energy use and happiness at multiple levels of spatial aggregation, and over time. To achieve this, we use the most comprehensive data available in
 spatial and temporal dimensions. The result is to  connect two largely separate fields: energy
 research and happiness research. This is the core contribution of this study.
 

%DO NOT SELF IDENTIFY IN THIS BLIND DOCUMENT
\noindent Best,\\
Authors
\vspace{.5in}

%  \rr{page/paragraph/line} {This is the format of text references}  \hspace{1.5in} 
% \vspace{.2in}


% Let me begin by thanking the anonymous reviewer for helpful
% suggestions. Below, I reply inline to your comments. The
% differences between last submission and this revison follow.

  %  XXXmake here any general remarks if needd e.g. abut use
  % of latex or specific approach/angle  i take in thsi paper that
  % explains why i proceeded in a certian way if reviewer did not see
  % thst--be constructive!!XXX


 
\newpage
\section{Response to Reviewer \#1} 


\cc{It is unclear that which regression models were used in the paper. The
  author(s) should describe the models in details ( fixed or random
  effects???). It should be noted that correlation analysis does not imply
  causal effects. How about other controlling variables such as corruption, social capital, religion, inequality}

Thank you for this suggestion -- we do think that it is a great idea. And we have supplemented our analysis with measures of religiosity and social capital. 
However, after investigation and to the best of our knowledge, we find that currently available data is not sufficient to measure these characteristics at the necessary level of aggregation for most of the cases included in this analysis.  
\\

More specifically, we added religiosity and social capital from the World Values Survey
(WVS). However, social capital variables are missing for more than half of the
sample. We also added corruption from the World
Bank and we evaluated multiple sources for Gini: World Bank Development Indicators, 
and finally found probably most comprehensive
\url{http://data.worldbank.org/data-catalog/all-the-ginis} which has recent
years available -- but unfortunately, most observations are missing. Deininger,
Klaus and Lyn Squire dataset is more complete, but  is unsuitable for integration with happiness data, since DKS provides coverages through the 1980s/early
1990s and happiness data with at broad spatial coverage starts from 2000s.
Since, the goal of this study is breadth,  we aim to cover as many
areas as possible. Taken together, addition of controls in these would cut sample size by
more than half. % We could still estimate regressions for say only European
% countries, but this study does not focus on Europe only.

Analysis of multiple predictors of happiness in addition to energy use is a
great idea, and we will pursue it in follow-up research because it would
 build on the current research, be more narrowly focuses. and be relevant to a somewhat different audience. Based on a preliminary analysis we judge that Western Europe  feasible domain for a followup study. 
\\



\cc{Energy use is endogenous, is likely to be affected by income per capita, how can the authors deal with this issue?}

We agree that endogeneity is important and we agree that energy use is potentially
endogenous. However, the existing available public data are not sufficient to support a full causal analysis. 
We have revised the text (primarily in discussing the results) to better clarify the limits of the correlational approach used, and to
suggest possible competing causal explanations. 

We believe that the correlational results in the present analysis are novel and interesting; and that they will contribute in a timely fashion
to the ongoing public around energy and happiness. 

We believe a causal analysis is appropriate for a later publication. How to conduct such an analysis, given the data available, is unclear. However, following \citep{sorensen12}, we would it may be possible to locate instrumental variables contained within data outside this area; or to discover an exogenous shock that serves as a natural experiment within a particular region. The study of energy consumption disruption appears intriguing: for example to look for cases where  energy consumption was disrupted by natural disaster, or power
 plant or electric grid failure; or to examine oil and natural gas disruptions--embargoes, shortages, dramatic price increases, etc. However, locating and evaluating appropriate instruments and/or natural experiments is a longer-term project. 

We have revised the text to make clear the limits of the correlational approach, and have eliminated the initial exploratory results using multivariate regressions that were contained in a supplemental appendix. \\

\bibliography{/home/aok/papers/root/tex/ebib}


\newpage
\section{Response to Reviewer \#2} 

\cc{This article addresses the relation between energy use and happiness by investigating the energy intensity of GDP and by analyzing the inter-state variation of happiness in relation to energy use over time. I am not sure about the added-value of the second issue, the first issue is worthwhile. However, because of the reasons described below, this article seem not reach the quality to be accepted by the JOHS.}


\cc{[Major points]
- Overall, many parts of the argument have been disturbed due to poor writing. It would be better to invite a senior scholar or someone familiar with writing articles in English.}

One of us is a senior scholar and a native speaker, and the manuscript has now been heavily edited.

\textbf{TODO Micah} %or Rubia or David Salas (he studies energy!)

\cc{- The relation between energy use and happiness seems very similar to the relation between economic growth and happiness (i.e. the Happiness Paradox). It is self-evident given that economic growth generally moves together with energy use. Although the author partially argues about this point and simply states that "finding the cause is left for the future research" (page 9, line 34) in the section of conclusion and policy implications, the connection between the finding and the Happiness Paradox would be better to be treated as central and investigated.}

Yes, indeed, this is a significant  point-- and we did not previously draw attention to its significance. We now emphasize it clearly in the meain text (see tracked changes), and have added related analyses in supplementary material. 

Notwithstanding, we believe that an analysis of energy use and happiness is a contribution of the research we conducted, and we retain the discussion of the relation between energy use and income.  the focus on energy use, and not income.

% Also, one reason is technical--there is overlap between
% observations and cross-sectional graph needs to be really big.
% Analytical part of paper is mainly about cross-sectional results and we put
% additional time-series graphs in supplementary online material
\\

\cc{--- Related to this, although the author analyses that "the highly-developed
  Nordic countries (DK, FI, NO, SE)" show "the positive relationship between
  energy consumption and happiness" (page 4, line 17-22), they are rather the
  cases representing the insignificance of energy use to happiness above a
  certain threshold.}

What we meant to communicate is that they lie at the top right of the first panel in figure 1, so that they contribute or drive the relationship having some leverage
on quadratic regression fit and making the relationship steeper or stronger. 
\\

After reexamining the figure in the light of your comments, we realize that while they
lie above the quadratic fit, they also lie below linear fit (except DK) -- so
indeed, you are correct that they signify a flattening out the bivariate relationship.

We have updated the text to describe this more complex relationship.

\cc{--- In contrast, the following sentence "At the country level, the lower the energy consumption, given development level, the happier the country." (page 8, line 46-48) seems contradictory with the overall argument of this article and the finding in Figure 1(1) in particular.}


We clarified the text by adding the reference to Panel 2 in Figure 1. We meant to communicate that while
 the bivariate relationship is positive (panel 1 of fig 1),  when taking
 account of income (panel 2 of fig 1) the relationship is positive.

\cc{- Although the author views that "With a notable exception of California, energy use is not decreasing" (page 8, line 34-36) in the US, it does not match with Figure 3. Indeed, 2 or 3 regions move downward, 2 or 3 regions standstill and only 3 regions upward. Nonetheless, this (i.e. downward and standstill trends in many areas in the US) seems not intuitive and needs explanation if the data are correct.}

Agreed. What we intended to communicate is that California has been the leader in
sustainable development. We have now clarified this in text -- in particular we call attention to the new england
and pacific regions as examples of regions that experienced declines.


\cc{- At the end of the article (page 10, line 53- page 11, line 5), the author
  introduces the impact of pollution on the relation between energy consumption
  and happiness but skipped the investigation simply by leaving the finding in
  the supplementary material. If it is introduced in the main text, it needs to
  be examined; otherwise, it would be better to drop it all together from the
  article.}

Agreed. We now realize this line of extension  of the main result was overly ambitious. We have eliminated it from the text, and plan to include it in a follow up paper.

\cc{[Minor points]
- "98\% of US emissions" (page 1, line 44) What emissions?}

carbon dioxide; fixed in text

\cc{- "a co-inventor of HDI, Amartya Sen, has proposed happiness as better
  measure of overall development or progress (Stiglitz et al. 2009)" (page 2,
  line 8-10). Sen is not a good reference here. Although he acknowledges the
  importance of happiness and subjective well-being to make informational space
  for evaluation broader, he stresses the significance of assessing well-being
  objectively in consideration of the risk of adaptive preferences. Given that
  the report was written by many contributors, it is most likely that this point
  was advocated by others.}


Agreed. In light of your comments we added a
footnote with clarificatio. We also replaced ``better measure'' with ``a measure''--as you are
probably right that Sen would still prefer ``more objective'' indicators. Nevertheless, we believe it is appropriate to retain a reference to Sen in text
because that paragraph discusses HDI, to which Sen is relevant.  
\\

We thank you for helpful comments--they have improved this paper, and
they will help us with follow-up papers on this topic.



\section{Tracked Text Changes}  
\textbf{(see next page)}

% use ediff to pull the latest revison and just save it as rev0
% or better: git show f370f9881d0dd450b2f6856824b3058e421da6bc:micah_eu_lr_welf.tex >/tmp/a.tex 
% (original state that was submitted) and then just:
% latexdiff rev0.tex free_from_to.tex > diff.tex
% pdflatex diff.tex [if want bib open in emacs and do usual latex/bibtex]
% and then 
%\includepdf[pages={-}]{diff.pdf}%don't forget to latex diff.tex!

\end{document}
%off

Added:
\begin{quote}
\input{/tmp/a1}
\end{quote}

%make sure that tags are in newline and at the begiiing!!!

sed -n '/%a1/,/%a1/p' /home/aok/papers/root/rr/ruut_inc_ine/tex/ruut_inc_ine.tex | sed '/^%a1/d' > /tmp/a1

sed -n '/%a2/,/%a2/p' /home/aok/papers/root/rr/ruut_inc_ine/tex/ruut_inc_ine.tex | sed '/^%a2/d' > /tmp/a2

sed -n '/%a3/,/%a3/p' /home/aok/papers/root/rr/ruut_inc_ine/tex/ruut_inc_ine.tex | sed '/^%a3/d' > /tmp/a3

sed -n '/%a4/,/%a4/p' /home/aok/papers/root/rr/ruut_inc_ine/tex/ruut_inc_ine.tex | sed '/^%a4/d' > /tmp/a4

sed -n '/%a5/,/%a5/p' /home/aok/papers/root/rr/ruut_inc_ine/tex/ruut_inc_ine.tex | sed '/^%a5/d' > /tmp/a5

sed -n '/%a6/,/%a6/p' /home/aok/papers/root/rr/ls_fischer/tex/ls_fischer.tex | sed '/^%a6/d' > /tmp/a6

sed -n '/%a7/,/%a7/p' /home/aok/papers/root/rr/ls_fischer/tex/ls_fischer.tex | sed '/^%a7/d' > /tmp/a7

sed -n '/%a8/,/%a8/p' /home/aok/papers/root/rr/ls_fischer/tex/ls_fischer.tex | sed '/^%a8/d' > /tmp/a8

sed -n '/%a9/,/%a9/p' /home/aok/papers/root/rr/ls_fischer/tex/ls_fischer.tex | sed '/^%a9/d' > /tmp/a9

#note has to be 010! otherwhise it picks a1!
sed -n '/%a010/,/%a010/p' /home/aok/papers/root/rr/ls_fischer/tex/ls_fischer.tex | sed '/^%a010/d' > /tmp/a010















\section*{Response to Reviewer \#2} 
\cc{In recent years an interest has developed in comparing quality of life on both sides of the Atlantic. This paper gives a refreshingly crisp report on why are there such marked differences in satisfaction with working hours in the US and Europe, a topic that according to the author is underrepresented in the QOL literature. The author pits cultural against economic reasons to explain preferences for longer and shorter working hours. An excursion into the value attached to work in the two study contexts suggests an explanation (see Table 2 and Figure 3). A major strength is the care taken to harmonise relevant data collected on both sides of the Atlantic.}

\rr{n/a} N/A

\cc{Quibbles:\\
Tables and graphs in the text are kept to a minimum. The remainder of the evidence is placed in the
appendices. This seems to work well.}

\rr{n/a} N/A


\cc {However, readers might like to see the questions contained in Tables 8 and 9 repeated in the
  legends to Tables 9 - 16, and Tables 18-20.} 

\rr{10-19//}  To make the appendices more concise I decreased spacing in tables from
1.5 to 1.0. As a result, there are now on average 3 tables per page instead of 2, and it is easier to
find them. Also length of the
manuscript decreased from  26 to 22 pages, which  saves space in the journal. Repeating questions in
legends will take up space and not clarify much: frequency tables are fairly self-explanatory.

\cc{The alignment of several items in Table 8 is incorrect and the wording of the item on the showcard for 'GSS friends' is missing: say: 'How often do you see your friends?'}

\rr{13//}  I fixed the alignment in Table 8 in column 1. And added ``Spend a social evening with friends who live outside the neighborhood?''


\section*{Reviewer \#3}

\cc{ This paper intends to explain the difference between Europeans and Americans
on work and happiness. It finds that Europeans work to live and Americans live to work. This is
indeed a cultural explanation which deserves attention of the academic field in the study of
subjective wellbeing.}

\rr{n/a} N/A

\cc{ In my view, the litmus test of this paper for acceptance of publication is
whether it uses the same measure of subjective wellbeing; as life satisfaction and happiness are
different, despite both are subjective wellbeing. The paper recognizes this difference and clearly
illustrates it in footnote no. 3, but it makes it clear that they are used interchangeably. This is
fine if two concepts are the same; however, when we go to the measurement -the Europeans were asked
about the extent of satisfaction "with the life you (respondents) lead"; so, this is a life
satisfaction measure. To the Americans, the question they were asked is the extent of happiness of
"things are these days";
so, this is a happiness measure. Therefore, the Europeans were asked about a cognitive judgment of
their life; a long period of time up to the moment they were interviewed. But Americans were asked
differently - "things are these days" indicates a shorter span of time and the idea is only about an
affective mood - happiness, without any cognitive evaluation as in the case of life satisfaction. In
other words, both concepts should not be treated as the same on the operational level in this
paper.}

\rr{5/2/6} This wording difference was acknowledged in the paper:

\begin{quote}
 Wording of the
survey questions is slightly
different (see  Appendix B), but these small
differences do not make surveys
incomparable. At least one other paper used the same surveys
to conduct successful comparisons
between Europe and the US (see \citet{alesina03}). 
\end{quote}

\noindent\citet[2013/2/11]{alesina03} defended this approach arguing that:
\begin{quote}
 ``happiness'' and ``life satisfaction'' are
highly correlated. 
\end{quote}


\noindent I reviewed recent literature and found another published paper  using the same survey items. \citet[211/2/20]{stevenson09w} defend this approach arguing that:  
\begin{quote}
While life satisfaction and happiness are somewhat different
concepts, responses are highly correlated.
\end{quote}

\noindent \citet{alesina03} \citet{stevenson09w} are able to make statements about correlations and compare
happiness with life satisfaction because there was happiness question in addition to life
satisfaction question in Eurobarometer until 1986
(still, they use life satisfaction measure because it is available for more years).
I use Eurobarometers in 1998 and 2001 (these are the only datasets with working hours available
for Europe) and I have to use ``life satisfaction'' measure. Both, \citet{alesina03} and
\citet{stevenson09w} use \underline{exactly the same survey items as this paper uses} to compare
happiness in the U.S. and Europe.

\rr{5/2/6} I have changed text FROM:

\begin{quote}
 Wording of the
survey questions is slightly
different (see  Appendix B), but these small
differences do not make surveys
incomparable. At least one other paper used the same surveys
to conduct successful comparisons
between Europe and the US (see \citet{alesina03}). 
\end{quote}

\noindent TO:

\begin{quote}
Wording of the
survey questions is slightly
different (see  Appendix B), but these small
differences do not make surveys
incomparable. At least two other papers used the same surveys
to conduct successful comparisons
between Europe and the US (see \citet{alesina03, stevenson09w}). ``Happiness'' and ``Life
Satisfaction'' measures are highly correlated. 
\end{quote}

\noindent Still, using different measures may be a limitation of this study, and this pertains to
independent variables as well. This limitation is acknowledged by adding  
footnote 5 on p. 5.

\begin{quote}
 Still, robustness of the results can be improved if wording of the survey
 questions is the same for all respondents. This remains for the future research when better data
 become available.
\end{quote}


\cc{Of course, a composite index combining both happiness and life satisfaction is a solution;
but unfortunately, this paper does not have it for this comparative study on subjective wellbeing
between Europeans and Americans. On the basis of this assessment, I do not recommend to accept the
paper for publication in its present form.}

\rr{n/a} If I understand this comment correctly, reviewer asks to combine happiness for the
U.S. with life satisfaction for Europe, but I believe this to be a misunderstanding: In order to
combine two different  measures into an index they need to be observed for the same
 individuals. This is not the case here: there is happiness for Americans and life satisfaction for Europeans.

\noindent The only way to create an index is to find both happiness and life satisfaction measures in the same
dataset, but they do not
exist. Again, this is not a serious limitation because happiness and life satisfaction
are highly correlated and the very same measures as used in this paper are successfully used in the
literature  \citep{alesina03, stevenson09w}.

