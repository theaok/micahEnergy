%to have line numbers
%\RequirePackage{lineno}
\documentclass[10pt, letterpaper]{article}      
\usepackage[margin=.1cm,font=small,labelfont=bf]{caption}[2007/03/09]
%\usepackage{endnotes}
\usepackage{setspace}
\usepackage{longtable}                        
\usepackage{anysize}                          
%\bibpunct{(}{)}{,}{a}{,}{,}                   
%\bibpunct{(}{)}{,}{a}{}{,}                   
\usepackage{amsmath}
\usepackage[pdftex]{graphicx}  %[pdftex]git latex doesn't like it             
\usepackage{epstopdf}
\usepackage{hyperref}                             % For creating hyperlinks in cross references


% \usepackage[margins]{trackchanges}

% \note[editor]{The note}
% \annote[editor]{Text to annotate}{The note}
%    \add[editor]{Text to add}
% \remove[editor]{Text to remove}
% \change[editor]{Text to remove}{Text to add}



\marginsize{1cm}{1cm}{.5cm}{.5cm}%{left}{right}{top}{bottom}   
					          % Helps LaTeX put figures where YOU want
 \renewcommand{\topfraction}{1}	                  % 90% of page top can be a float
 \renewcommand{\bottomfraction}{1}	          % 90% of page bottom can be a float
 \renewcommand{\textfraction}{0.0}	          % only 10% of page must to be text

 \usepackage{float}                               %latex will not complain to include float after float

\usepackage[table]{xcolor}                        %for table shading
\definecolor{gray90}{gray}{0.90}
\definecolor{orange}{RGB}{255,128,0}

\renewcommand\arraystretch{.9}                    %for spacing of arrays like tabular

\newenvironment{ig}[1]{
\begin{center}
 %\includegraphics[height=5.0in]{#1} 
 \includegraphics[height=3.3in]{#1}
\end{center}}

 \newcommand{\cc}[1]{
\hspace{-.13in}$\bullet$\marginpar{\begin{spacing}{.6}\begin{footnotesize}{#1}\end{footnotesize}\end{spacing}}
\hspace{-.13in} }

\usepackage{datetime}


%\usepackage[latin1]{inputenc} %git latex compiler doesn't likeit
\usepackage{tikz}
\usetikzlibrary{shapes,arrows,backgrounds}


%\usepackage{color}					% For creating coloured text and background
%\usepackage{float}
\usepackage{subfig}                                     % for combined figures

\renewcommand{\ss}[1]{{\colorbox{blue}{\bf \color{white}{#1}}}}
\newcommand{\ee}[1]{\endnote{\vspace{-.10in}\begin{spacing}{1.0}{\normalsize #1}\end{spacing}\vspace{.20in}}}




\usepackage{sectsty}
\allsectionsfont{\normalfont\sffamily}



\usepackage{sectsty}
\allsectionsfont{\normalfont\sffamily}
%\usepackage[margins]{trackchanges}

\renewcommand\familydefault{\sfdefault}

\usepackage{verbatim}
\usepackage{rotating}
\usepackage{catchfilebetweentags}
%-------------------- END extra options -----------------------------------------
\date{Draft: {}\today}
\title{
%  The Paradox of Energy Consumption and Happiness Across Countries
%Presonal 
Energy Use And Happiness\footnote{Author Contributions: A.O.K and M.A designed
  research. A.O.K. performed research. A.O.K. analyzed data. A.O.K. and
  M.A. wrote the paper.  % Micah: Add author contribution statements. See: http://www.nature.com/news/publishing-credit-where-credit-is-due-1.15033
}
}
\author{
Adam Okulicz-Kozaryn\thanks{EMAIL: adam.okulicz.kozaryn@gmail.com
  \hfill I thank XXX.  All mistakes are mine.} \\
{\small Rutgers - Camden}\\
Micah Altman\thanks{EMAIL: ???
  \hfill I ???} \\
{\small MIT}
}

\begin{document}

%%\setpagewiselinenumbers
%\modulolinenumbers[1]
%\linenumbers

\bibliographystyle{/home/aok/papers/root/tex/pnas2011.bst}
\maketitle
\vspace{-.4in}
\begin{center}

\end{center}

\begin{abstract}
\noindent  


It is widely claimed that there is a substantial tradeoff between energy
 preservation and wellbeing. Cutting consumption would decrease wellebing. 
Despite technological advances, the Earth per capita energy use continues to grow.
The environmental consequences of high rates of energy consumption are well known:
resource depletion and pollution. Is this avoidable?

We study the relationship between energy consumption and happiness across four decades, and across multiple levels of geography.  Surprisingly, we find that received wisdom is false -- for counties, states and nations, energy consumption is neither necessary for well-being, nor linked directly to it. 

%aok: i though we focus now too much on en intensity, and this is just one panel
%of one graph; the overall argument was about energy use in general

% We find instead that national well-being is strongly associated with the energy intensity of GDP. Although not a necessary condition for well-being, countries that efficiently convert energy to economic wealth, may benefit from increased energy use. 

% We lay out the possible causal explanations for differences in the energy intensity of GDP. Based on this we summarize the potential implications for designing policies that yield both high well-being and modest energy use. 
\end{abstract}


\vspace{.15in} 
\noindent{\sc keywords:  energy use,  energy intensity of economy, sustainability, wellbeing,  happiness, Subjective Well-Being (SWB)}
%\vspace{-.25in} 

\begin{spacing}{1.4}
\rowcolors{1}{white}{gray90}

%  \ExecuteMetaData[../out/tex]{ginipov}

% \begin{figure}[H]
%  \includegraphics[height=3in]{../out/gov_res_trust.pdf}\centering\label{gov_res_trust}
% \caption{woo}
% \end{figure}

% Micah: 
%  Start with the strongest introduction we can possibly support. If we can't rigously support it with the data and analysis, we'll weaken it... but this is the goal. 
%  -- begin intro -- 

Environmental consequences of human consumption are the largest challenges to
science and society today. Energy consumption is a key component--the climate problem is mostly an energy problem \cite{mackay08}. Despite
technological advances, the Earth per capita energy use has increased about 40\%
over past 40 years and it continues to grow. % (1890-1336)/1336
Most energy consumption both pollutes and  depletes natural resources
\cite{arrow04, soytas07}. %recent us: http://www.eia.gov/tools/faqs/faq.cfm?id=427&t=3 and
                          %for world even worst http://www.google.com/url?sa=t&rct=j&q=&esrc=s&source=web&cd=7&sqi=2&ved=0CDwQFjAG&url=http%3A%2F%2Fwww.iea.org%2Fpublications%2Ffreepublications%2Fpublication%2Fkeyworld2014.pdf&ei=uPMVVYaSM8OhgwSjqIKIAQ&usg=AFQjCNFX92MbI8lsvnDKqHCZqqoBnSMtSQ&sig2=NZpGqw3hDWdm3xcxvQyizA&bvm=bv.89381419,d.eXY&cad=rja  
 Energy consumption has, of course, many benefits as well.
 % TODO check these
% references if they are really to the point ! and possibly add some from goog
% drive folder
% Energy use and pollution are highly correlated, but they are not the same,  and
 The question remains about the net effect, how energy use affects human wellbeing.
 % we will differentiate between various types of energy use.  
We use a happiness yardstick
to evaluate benefits and problems of energy use. Traditionally, Gross Domestic
Product and its adjustments--per capita and purchasing power parity--have been used to evaluate development. Human Development Index (HDI)
added life expectancy and education, but more recently a co-inventor of HDI,
Amartya Sen, has proposed happiness as better measure of overall development or
progress 
\cite{stiglitz09al}. 

% importantly there is spatail mismatch in resource demand and supply--notably
% china need a lot and has few; opec countries have oil; etc etc
%for instance recent emissions cap deal between US and China has largely to do
%with enrgy use e.g. http://mobile.nytimes.com/2014/11/13/opinion/climate-change-breakthrough-in-beijing.html?_r=0
It is universally acknowledged that there is a fundamental tradeoff between
societal energy preservation and individual self-interest. Substantially reducing
energy consumption requires individual sacrifices. If we reduce consumption, our wellbeing will
suffer \cite{kenny_businessweek_aug_29_14, gordon_wsj_may_29_14, carter_pbs_apr_18_77,smil05}. 
%COPIED FROM http://www.sciencedirect.com/science/article/pii/S0301421511001042
% The remarkable improvements in quality of life that occurred during the industrialization of Europe, North America and Japan in the 19th and early 20th centuries were caused, in large part, by the invention and adoption of energy intensive technologies (Smil, 2005). Coal was used to fuel steam engines, petroleum fed internal combustion engines, and all the fossil fuels plus falling water turned electrical generators.
%
%also, the point is that there is more popular wisdom instead! we know little scientifacally!hence our paper!
%
% and there is a Lancet series; but they did with health what we do with
% happiness:energy helps with health but also destroys it:
% http://www.sciencedirect.com/science/article/pii/S0140673607612537
% http://www.sciencedirect.com/science/article/pii/S0140673607612586
% http://www.sciencedirect.com/science/article/pii/S0140673607612525
% http://www.sciencedirect.com/science/article/pii/S0140673607612598
% http://www.sciencedirect.com/science/article/pii/S0140673607612550
%
%
% A very fundamental question is how to achieve both happiness and
% sustainability--it is difficult because there appears to be a tradeoff.
% Fundamentally,
%  the ultimate goal of increased energy consumption is human
%  happiness--if there is no relationship between the two, then we can
%  consume less and stay happy.
%
In this paper, we find that this universal assumption is wrong.  By combining
data on energy consumption and happiness we find that the
relationship is weak at best. % , with exception of poor countries, where more energy is% nneded for greater happiness.
 People in areas consuming more energy are not happier.
This finding is robust across time, and multiple levels of spatial aggregation--it applies to patterns of energy consumption  at the local, national, and global scale.  


% -- end intro -- 

%Energy consumption matters.% beyond happiness
 Energy is a strategic
resource. Countries wage wars about energy sources  and much of politics is
driven by energy. Many countries' whole economies depend on energy exports, for instance OPEC countries. Many other countries rely heavily on energy production
and some use it as a political tool, for instance, Russia.   
Virtually all countries  always seek to obtain more energy sources. A recent example is so called
fracking. Yet, we need to consume less. There are at least two
obvious reasons--most of energy consumed  is non-renewable and will stay
that way for some time \cite{mackay08}--most energy comes from
fossil fuels, which will run out. %MAYBE be specific give numbers
Second, energy consumption results in pollution, and pollution harms not only
 environment and other species, but also  humans \cite{mackerron09,gandelman12,ferreira13}, the beneficiaries of energy
consumption. Pollution potentially cancels out the benefits of energy
consumption. 
 
A recent report by
Intergovernmental Panel on Climate Change is alarming 
(\url{http://www.ipcc.ch/}). % very important body; another pnas citing it
                             % already in title http://www.pnas.org/content/104/24/10288.full
Indeed, a threat is serious enough that
claims for not growing the economy anymore or even ``degrowing'' it
appear reasonable \cite{kallis11, kallis12}. % \footnote{Although it is not
  % immediately obvious what is the best strategy to achieve greater
  % sustainability--degrowth \cite{kallis11} or simply public policy
  % \cite{bergh11}--for more discussion see \cite{daly13,kallis12}.}
At very
least curbing consumption is a reasonable course of action. Some argue reduction as high
as by factor of 10 in affluent societies \cite{pretty13}. This,
of course, begs a question, what would happen to our wellbeing?
 Again, a common % or received
wisdom is that there is a link between happiness and
energy use--we need to consume energy in order to achieve greater
happiness. One could even say, that the very end goal (usually implicit)
of energy consumption is wellbeing or happiness.  % This is
% true in buivaraite case--the richer areas are happier.
%  We argue here that it is not necassarily so, only after taking into account
%  income at country level % and few other variables at state or county levels,
%  the relationship between energy consymptiona nd happiness disappears.
%
% its striking 2 fold or more differences between states like nj and tx and only a
% small fracton of that is due to climate most due to consumption arguably
% wateful/conspicous--calculate how mich explained by climate! give number
Arguably a major, if not key, factor prohibiting us from conservation and
sustainability is fear of loss in wellbeing--we argue here that such loss, if
any,  will
not be substantial. Indeed, our results suggest that curbing energy consumption that results
in pollution, i.e. most of today's energy use, will not affect adversely our
happiness in the developed world.  % There is anectodal evidence pointing to the fact that humans can
% live happily without much energy--among Amish, African Maasai, and Greenlandic
% Inughuit, most people are above neutral in well-being. \textbf{TODO cite!}

At the country level, we show that the lower the energy consumption, given development
level, the happier the country.  Across US states and California counties,
energy consumption and happiness have a nil relationship. 
% it would be another reason for energy
% conservation.
Likewise, the over time changes in energy use are almost unrelated to changes in
 happiness across US Census regions. 
These are important findings  because many assume that energy consumption makes us
happy--this is presumably why we keep on consuming ever more energy and are
reluctant to curtail this consumption. % At the same time, there are important
% reasons to curtail the consumption.
With this correlational study we aim to bring the relationship between energy
use and wellbeing to wider audience, and encourage more research in this area. 

% Germany plans for renewable energy
% http://www.nytimes.com/2014/12/01/business/energy-environment/plan-outlines-low-carbon-future-for-germany-energy.html

\section*{\large \bf Results}

% Micah: This needs to be reorganized. 

% In my view the primary goal of this section is to answer this question:
%  *** Which choices do individuals make, that have a large negative effect on the environment, and which do not make the individuals, their community, or society better off?  ***
% We aim to provide the best scientifically and empirically supported answer possible, that does not require collecting new primary data. 


% AOK: another paper !
% this is interesting, probably more interesting than what we have so far, the
% problem is that we did not really test for that; now i was actually surprised
% that there are  data by census or climate
% regions (9 of them): 
% http://www.eia.gov/consumption/residential/data/2009/index.cfm?view=consumption
% and we have happiness there too: but you are asking a grand question; and the
% data we have just breaks down energy use in a big region estimating how much is
% used for cooling, heating, lights, etc etc--but there are a function of climate,
% culture and other collective stuff; for your question we would need person-level
% data and there are some (PSID for instance), but that's a different paper...

% what we have done so far is that there is not much happiness from energy use;
% probably much less than expected...

% and then we can say that given that we can cut energy consumptin and stay happy

% so yes in a sense we do answe your question--it is just not ``which choices''
% buty specifically less energy in general

% I suggest the following organization
%  1. Provide an easy to interpret figure showing happiness vs. energy consumption at the global level (nations), for a single time period. This should illustrate the general weakness of the relationship between happiness and energy consumption without controlling for anything. Discuss the substantive size of the raw - uncontrolled correlation
% 2. Now discuss the details of the measures. explain that (a) the energy
% consumption measures include only those that are under personal control, 

% AOK: all residential is personal control; but affected by climate etc--hence we
% have controls

% and (b) have a substantial effect on environment. Cite the literature as appropriate

% AOK ok, cool will trhow in lit here

% 3. Show that this relationship is durable -- introduce time-series data to show that this holds over significant time-scales

%AOK:by relationship i guess you mean happiness-energy use; but happiness is
%mostly flat over time; same energy use: in the rich countries mostly flat or
%increaing a little; in developing it increases;so won't be much there

% 4. Show that this relationship is not an accident of scale. Drill down to states, then counties within california, and show the relationship holds
% 5. Now discuss other possible explanations for these patterns (a) use these explanations to suggest convariates/confounders (b) show the results of controlling visually first, or with a simple table, (c) use a statistical model to ensure that the visual impressions are not misleading

\textbf{Global patterns.} We start by examining the relationship between energy use and well-being across the world, at the level of individual countries.  % --there is not much variation in happiness over
% time, and energy consumption in developed areas is quite flat, too. 
We use happiness data from World Database of Happiness
(WDH).  This happiness dataset is
based on multiple sources (for details see \url{http://www1.eur.nl/fsw/happiness/hap_nat/nat_fp.php?mode=8}).
  Happiness is mostly measured with answers to
"All things considered, how satisfied are you with your life as a whole these
days?" and data are expressed on scale from 0="dissatisfied" to 10="satisfied." 
The measure of total energy  consumption comes from the World Bank
(\url{http://data.worldbank.org/indicator}). Data are plotted in figure \ref{couWvsLsEnePerGdp2}.
 %we use total and not residential
                                %electricity because across countries it is
                                %about energy intensive technologies of the
                                %whole economy, type of economy...,  not just households that
                                %determine QOL and happiness of people; within
                                %countries these ar emore or less constant and
                                %hence we can focus more on personal consumption
% 
%
% may also say that there are severalfold diff not only in income but also in ene

The basis for the received wisdom that increasing well-being requires increased
energy consumptions is illustrated by the figure on the left: Across countries,
there is a clear positive association between greater energy use, and greater
well being. Although the relationship between
well-being generally increases with energy consumption -- the variance across
countries is tremendous. Further, some countries with high energy use, such as
Russia (RU) have low overall well being, the highest energy
users are not the most well-off overall, and many countries such as Costa
Rica (CR) or  Mexico (MX) are able to reach the highest level of subjective well-being while maintaining very low energy use. What could account for this? 

% 4/13
% TODO 
%
% Fig 1a - graphical emphasize high well-being/low energy countries
% Fig 1a - do we really believe Mex, Guatemal and Colombia have higher well-being -- seems like WVS may be incompletely
%		   measuring this

Moreover, the three countries that strongly contribute to the positive
relationship between energy consumption and happiness are the highly-developed
Nordic countries (SE, NO, FI) -- which have both unusually high energy use and happiness. Prior work shows that higher development is related to greater quality of life and well-being \cite{mazur11}: Could the relationship between energy use and well-being be driven by characteristics of development unrelated to overall energy consumption?

% 4/13
% TODO 
%
% Is the attribution to Nordic countries correct?
% Can we answer the question, at least partially?
% sure there is a lot of other things predicting happiness! Scandinavia uses a
% lot of energy because it is freezing there, and it is arguably v happy due to
% great governance, welfare, nature, low density etc etc

Most surprisingly,  we find that when we measure energy as as a function of economic efficiency, the relationship reverses.  This is shown in the second panel, which displays, on the x axis, the energy intensity of the gross domestic product (energy/GDP). This reveals that countries that  consume less energy per unit of wealth are better off. 

In a descriptive sense, high energy intensity indicates that a country requires
a high cost to convert energy into GDP. Another way to put this is that some
countries are more efficient (in the economic, not technical sense) at
converting energy into wealth. There are a number of possible causal explanations. One possible cause that
happiness is higher for countries with lower energy intensity of GDP is the use
of more energy-efficient technologies--these technologies reduce energy inputs
for the same output of goods. Importantly, consumption can remain high at low
levels of pollution if there is not much energy used per GDP.
% can elaborate a bit if needed better efficiency--less pollution per consumption--more good stuff without the bad stuff

%was thinking about how to neutralize too happy COL and few others, but it is
%what it is,may explain this a bit in text perhaps--e.g. see http://www.huffingtonpost.com/2013/01/17/reasons-colombia-happiest-country_n_2490813.html
In figure \ref{couWvsLsEnePerGdp2} most countries conform to the predicted pattern. 
 For instance,  the Netherlands (NL) is rich,  energy efficient and happy. But
 there are also some outliers, for instance,  Colombia (CO) is  happy and energy
 efficient,  but poor.  
In general, Latin America poses a puzzle for happiness researchers. Latinos are
relatively poor and yet quite happy. They also use very little  energy and have similar energy intensity of GDP to that of the  US. Notably, all Latin
 American countries cluster at the top left in the
first panel.  Great happiness is possible using little  energy. East European
post-Soviet countries, on the other hand, cluster at the bottom and some at the
left. Some countries are relatively unhappy despite low energy intensity of GDP,
such as Albania (AL) and Congo (CG). Some countries, on the other hand, are relatively happy
despite low energy efficiency such as Trinidad and Tobago (TT) and Kazakhstan (KZ).    
% also see http://www.ritholtz.com/blog/2010/06/oil-consumption-around-the-world/ 
\begin{figure}[H]
 \includegraphics[width=6in]{graphsAndTables/couWdhEneGdp.pdf}\centering \caption{Happiness against GDP,  and energy intensity of GDP. Quadratic fit shown with 95\% confidence intervals. Happiness data come from World Database of Happiness and is measured on scale from 0="dissatisfied" to 10="satisfied.'' Data come from multiple sources and were averaged for 2000-2009 period. For details see \url{http://www1.eur.nl/fsw/happiness/hap_nat/nat_fp.php?mode=8}.  Energy use refers to use of primary energy before transformation to other end-use fuels, which is equal to indigenous production plus imports and stock changes, minus exports and fuels supplied to ships and aircraft engaged in international transport. % For average electricity consumption per electrified household, which shows a similar picture, see figure \ref{couWvsLsEleHHgdp} in supplementary material.
For graphs using happy life years as a dependent variable see graph \ref{hly} in
supplementary materials.  Note: Country codes are in table \ref{ls} in supplementary material. Energy and GDP data were also averaged over 2000-2009 period. In addition several outliers were dropped: countries with energy use above 10,000: United Arab Emirate Iceland Kuwait Qatar Trinidad and Tobago; and countries with energy intensity higher than 2: Ethiopia Turkmenistan Uzbekistan.}\label{couWvsLsEnePerGdp2} 
\end{figure}
% \begin{figure}[H]
%  \includegraphics[width=6in]{graphsAndTables/couWvsLsEnePerGdp2.pdf}\centering \caption{Happiness
%    against GDP, total energy use, and energy intensity of GDP. Quadratic fit
%    shown with 95\% confidence intervals. Happiness is measured with "All things considered, how satisfied are you with your life as a whole these days?" on scale from 1="dissatisfied" to 10="satisfied." Energy use refers to use of primary energy before transformation to other end-use fuels, which is equal to indigenous production plus imports and stock changes, minus exports and fuels supplied to ships and aircraft engaged in international transport. For average electricity consumption per electrified household, which shows a similar picture, see figure \ref{couWvsLsEleHHgdp} in supplementary material.  Note: Country codes are in table \ref{ls} in supplementary material. We use a cumulative file covering first four waves, 1981-2007. If country was observed in more than one year, data were averaged.}\label{couWvsLsEnePerGdp2} %couWvsLsEnePerGdp2 doesn't have gdp only; couWvsLsEnePerGdp2 adds gdp as first panel
% \end{figure}


% In poor countries more income and sometimes more energy may be beneficial to human
%  wellbeing. In rich countries, on the other hand,  it is likely that not only we do not need
% more energy, but also, arguably we should consume less energy \cite{mazur74}. There are even
% calls to stop growing income in rich countries or even degrow it
% \cite{kallis11,kallis12,bergh11}.
We  focus now on the US in an effort to answer
an old question of whether more energy is needed to increase wellbeing if there
is already a great deal of energy being consumed \cite{mazur74}. The US is among
countries using most energy per capita.
% Also, we focus
% now on electricity consumption because it is representative of modern
% energy %niu13 
% and it is growing fast. %winfrey13 p7
% electricity for CA only--can add this explaation + lack of total energy use i
% guess if reviewers compalin about electricity in california! 
%
 We zoom in on US states and California counties.

State and county level happiness data come from Behavioral Risk Factor
Surveillance System (BRFSS) using a very similar question to WDH ``In general,
how satisfied are you with your life?'' on scale 
from 1=''very dissatisfied'' to 4=''very satisfied.'' State energy use data come from
U.S. Energy Information Administration and is measured as  total energy
consumption per capita in the residential sector.  
California's  residential electricity consumption per capita come from 
Energy Consumption Data Management System
(\url{http://www.ecdms.energy.ca.gov/elecbycounty.aspx}). Results are shown in
figure \ref{stateCaPAP}. Energy-hungry states (with possible exception of transportation sector) are not happier% --this is just another argument for energy conservation
. There are two outliers, Hawaii and California, consuming much less energy in
the residential sector than others. 

We zoom in on California counties in
second panel of the same figure. Like among states, there is a great deal of variation in energy
use across California counties, and also the 
relationship with happiness is nil. 
% So California consumes only about half of the energy consumed by an average
% state.  also see http://www.skepticalscience.com/print.php?n=1365
We have also experimented with energy intensity of
GDP as we did earlier across countries, but in case of the US subregions the results are
not different. %supplemnetart material picture XXX

%MAYBE can make it flatter as fig 1 so that subgraphs are square
\begin{figure}[H]
 \includegraphics[width=6in]{graphsAndTables/stateCa.pdf}\centering
\caption{Happiness and total residential energy use across US states and
  residential electricity use across   
  California counties. Happiness is measured with ``In general, how satisfied
  are you with your life?'' on scale from 1=''very dissatisfied'' to 4=''very
  satisfied.'' State energy use data come from
U.S. Energy Information Administration and is measured as  total energy
consumption per capita in the residential sector.  
California's  residential electricity consumption per capita come from 
Energy Consumption Data Management System
(\url{http://www.ecdms.energy.ca.gov/elecbycounty.aspx}).
We have also tried total energy consumption and its GDP
  intensity and relationship was also quite flat--see figure \ref{lfTETPBgdpLS}
  in supplementary material.}\label{stateCaPAP}
 \end{figure} % {\scriptsize Note: Country codes are in table \ref{ls} in
              % supplementary material. If country was observed in more than one
              % year, the data were averaged. }%PNAS readership should know US
              % state codes!

{\bf Over Time Movement.} It is well-known that happiness is related to income
in  a 
cross-section, but not over time \cite{easterlin74,easterlin12}. We
supplement our cross-sectional explorations with a look over time. We use General
Social Survey data.   %should be representatve %of regions but couldnt find any specific %evidence of that!
Happiness question reads "Taken all together, how would you say things are
      these days--would you say that you are very happy, pretty happy, or not
      too happy?" 1=''not to happy'', 2=''pretty happy'', 3=''very happy''. 
 Energy consumption is measured as total energy use per capita in
residential sector, the same measure as used for states in last section. 
Figure \ref{cenDivLsYrSm} shows happiness over time by
census division. There is not much co-movement. The two series correlate at
.2 only. In short, neither in
cross-section, nor over time energy consumption is related to happiness in
the US. 

\begin{figure}[H]
 \includegraphics[width=5in]{graphsAndTables/cenDivLsYrSmINKSCAPE.pdf}\centering
\caption{Happiness (6-yr moving average) and total energy consumption
  in residential sector per capita. Correlation is .2 only (for
  unsmoothened series) for unsmotthened happiness see figure \ref{cenDivLsYr} in
  supplementary material. Across countries there is not much variation in
  happiness over time, neither in energy use--see figure \ref{ebTS} in
  supplementary material. }\label{cenDivLsYrSm}
\end{figure}

Americans continue to consume large amounts of energy as compared to other
countries.  With a notable exception of California, energy use is
not decreasing, and in some cases energy use has increased. 
Also, Americans do not spend any less  on energy either--5-10\% of personal
expenditure is spent on energy over past 50 years, and if anything this amount
has  increased slightly  over the last decade \cite{bea-2-8-5}.
%   \url{http://burnanenergyjournal.com/what-goes-down-steins-law-and-the-cost-of-energy/}
% or \url{http://environmentalresearchweb.org/blog/energy-the-nexus-of-everything/}


\section*{\large \bf Discussion}

% Micah: We'll rewrite this later. The main goal is to (a) briefly summarize the most significant results; (b) outline the possible causal explanations and frameworks, and thus define the future research agenda; (c) describe potential interventions 

% 4/13
% TODO
% This jumps traight to causal conclusions.
% Need to be very careful about evaluating the evidence for the explanations we propose:
%   - energy efficiency 
%   - hedonic treadmill
% 	- conspicuous consumption (not the same as treadmill)
%	- positional good consumption
%	- risk-homeostasis based consumption (bigger cars)
%	- comparative national advantage in energy-intensive industries
%	- different baseline requirements for climate 
%
% For each of these explanations
%    - can we rule them out (or is their strong evidence against them) in the current data
%	 - what type of observation would be needed to test them (what level of aggregation of people, type of energy use, over time, an do we need natural expderiment)
%	 - what are appropriate interventions
%	 - which  of the interventions increase well being w/out increasing energy use, using current current feasible tech?
%	 - which of the interventions can increase well being w/out increasing energy use, requiring tech?

Happiness can be achieved at low levels of
energy consumption. At high level of energy consumption, such as
that in the  US, energy and happiness have a nil relationship. % If developed world changes from society of consumers
% to society of conservers, we will not suffer much in terms of
% happiness, if at all.
At the same time, however, it is widely assumed that the relationship is
positive. This discrepancy between expectation and experience can be illustrated
  in figure \ref{fUT}. A decision on how much
energy we would like to  consume is based on expected happiness. But we are
often (predictably) wrong and experience much less happiness.

\begin{figure}[H]
  \begin{centering}
    \includegraphics[height=2.0in]{graphsAndTables/utility}
    \caption{Expected vs. experienced  happiness \cite{kahneman97ws}. We make
      a decision about consumption based on expected or
      decision happiness, but the experienced or true happiness is
      lower when we achieve greater consumption.}\label{fUT}
  \end{centering}
\end{figure}


While in poorest countries, more  income and perhaps more energy
 is needed in some cases to increase happiness; in developed world, we arguably could  decrease
energy consumption without much loss in happiness. % can alse give some numbers on how much energy uses an american v
            % chinese etc
Perhaps, Indians and Chinese need to consume more energy, but not Americans. 
Texans could consume as much as Californians (they consume more than twice as
much) and no tragedy would happen. %http://www.eia.gov/state/rankings/
% Furthermore, it is unfair for the developed World to lecture developing
% countries about energy conservation without first making very substantial cuts
% at home.

Why energy use is not associated with happiness in the US? 
Consumption beyond a point does not buy happiness. Such consumption buys
position, but because everybody tries to oucompete everyone else, this race
cannot be won. A related cause is a ``hedonic treadmill'' \cite{brickman78cj}. Increased consumption does not necessarily increase well-being due to hedonic-adaptation. In simple words, more stuff
doesn't make people happy if basic needs are already satisfied. Hence, positive
relationship between happiness and energy across countries, but no relationship
in the US. % --poor countries need to consume more enrgy in order to increase their
% happiness, but not the rich countries
 A related explanation can be made using Veblen's concept of conspicuous or
 wasteful consumption \cite{veblen05a, veblen05b}. Such consumption does not
 satisfy needs but simply aims to demonstrate that
one is better than others. Much of such consumption wastes energy without gain
in wellbeing. % , for
% instance, mansions and large cars.  Many other examples are not necessarily conspicuous, 
% but they waste energy and have no clear contribution to lasting happiness, for
% instance, much of landscaping including unnatural ponds, fountains, and other items that
% define American suburbia.

Many social scientists have suggested that conspicuous consumption
does not make us any happier  \cite{csikszentmihalyi99, frank04, frank05, frank12}, but few actually test it. We contribute with a quantitative study of
the relationship between energy consumption and happiness. % and again, the only  study testing the
% effect of energy use on happiness was \cite{graef81}.
 Results suggests that  energy consumption can be substantially reduced
without making people less happy.  Now we turn to  possible interventions that could reduce energy consumption
 relatively painlessly. 

%NOT SURE IF WE NEED THIS
%  Similarly, increased income at country level does not lead over time to increased happiness--the so called Easterlin paradox
% \cite{easterlin74,easterlin10B}.  It is possible that some countries are more efficient because they have avoided this treadmill effect.
% Neither do conspicous consumption nor positional consumption result in
% happiness. There is more happiness gained from buying experience than buying
% things. For instanace, bowling, yoga, gardening result in more happiness than
% extra bedroom or larger house.  For most people,
% house is the biggest purchase they ever make and Americans tend to
% prefer big houses, but to afford them they need to commute longer
% distance. Humans adapt to big houses, but not to commute \cite{stutzer03,kahneman04,frank05}. 
%  Notably, both large houses and commute use energy that does not translate to happiness.  
%  Any luxury consumption (not only large houses) % SUVs, LV handbags
%  is not likely to result in lasting happiness because luxuries  are positional goods--people buy them to have a better position in a
% society, or to show that they are better than others--not to have a better quality of life.  The problem with
% positional goods is that acquiring them leads to consumption
% arms-race. You cannot win--you are  on hedonic treadmill.
% --we compare to others and, we
% adjust \citep{michalos85}
%  Due to this arms-race we ended up with
% ridiculously  big and expensive McMansions, SUVs, and other building
% blocks of American suburbia.

% What is the mechanism? Why energy consumption is not likely to buy much
% happiness?  There are several psychological explanations about causal pathways.
% We should buy experience, not things. Experience consumption results in more
% happiness than consumption of things. 
% That is  probably why  energy
% consumption in transportation brings most happiness out of different types of
% energy consumption, because much of
% transportation is about experience, for instance, travel, vacations. 
% % TODO from ls\_car lsPol and: 
% % Here psychological literature helps.
% Simply material consumption (where arguably most of the  the energy consumption
% falls) does not result in lasting happiness as opposed to experience
% consumption. Indeed, sustainable consumption does not necessarily mean less
% happiness, and there are many examples of activities low in carbon footprint but
% high in happiness such as gardening or yoga \cite{madjar06}. There is also
% consumption arms race that increases consumption, but not happiness. People are trying to outcompete each other in
% consumption but such race cannot be won and accordingly it does not result in lasting
% happpiness \cite{frank12}.% MAYBE hedonic adaptation etc etc
% % Finally,  pollution--energy cnsumption and $CO_2$ emissions correlate at above
% % .9, and pollution reduces happiness. %TODO cite


%------HOUSING guess irrelevant----------------------------
% One implication of the study is for housing: house is most expensive consumption
% item most people buy. Furthermore it costs considerable energy to build and
% maintain. The median house size in 1973 was 1,525 sq ft, and 2,169  sq ft in 2010.\footnote{\url{http://www.census.gov/const/C25Ann/sftotalmedavgsqft.pdf}}
% Surely, the families must have gotten bigger over that time. Well,
% actually they gotten smaller--in 1973 the average household size was
% 3.01, which dropped to 2.61 in
% 2010.\footnote{\url{http://www.census.gov/population/socdemo/hh-fam/tabHH-6.pdf};\url{http://quickfacts.census.gov/qfd/states/00000.html}}
% The problem with big houses is that it is never big enough anyway--we
% compare it to other houses, and size is an obvious metric. But others
% do the same and they get ever bigger houses all the time so we all end
% up with bigger houses but nobody is happier \cite{frank12}. ``A house may be large or small; as long as the neighboring houses are
% likewise small, it satisfies all social requirements for a
% residence. But let there arise next to the little house a palace, and
% the little house shrinks to a hut'' (Marx and Engels 1849, quoted in
% \cite{dittmann10}). \cite{dittmann10} find support for the above
% statement. \cite{luttmer05} also finds that the richer the neighbors,
% the less happy the person.
% \cite{firebaugh09} refine this relationship by confirming that people
% are happier in poor counties, but in  rich neighborhoods. 
%========================================================================

Two interventions to decrease energy consumption are suggested. First, we simply
need to increase awareness of what we have just described--that increasing
already substantial consumption does not buy much happiness, if any. Just as increasing  income
 beyond a point does not result in much happiness \cite{kahneman10}, increasing
 energy consumption beyond a point does not result in much happiness
 either. % as already suggested earlier \cite{mazur74, mazur11}
  In short, happiness can be achieved at low
 levels of energy consumption--human flourishing requires energy to satisfy
 basic needs only. Second, we simply recommend higher taxes on non-renewable
 energy to lower its use. 

It is striking that there were only three  attempts to
relate energy consumption to happiness.   % in 70s proxied 
 In a small sample of 55 countries, one early study used a set of 27 indicators to measure
 quality of life \cite{mazur74}, which was later extended over time \cite{mazur11}.
 A third study is a recent master
 thesis that  analyzes cross national data \cite{winfrey13}. But neither
 study  explores energy intensity of GDP, nor variation at finer geographic representation than a country.
 % We know that pollution makes
% us unhappy \cite{mackerron09,gandelman12,ferreira13}.
Furthermore, we contribute with a finding that energy consumption does contribute to happiness, if pollution is
taken into account. Results are in the supplementary material. This is a clear argument in favor of clean energy--if we can
consume energy without pollution, it should result in greater happiness. Until
we can to do so for vast majority of our energy needs, we should curb the energy consumption and
our wants or desires.

There is a need for future research in this important area. There have been many
calls to systematically collect happiness data, and we should collect energy use data for the same persons. Such data would allow
to explore the relationship better at person level. There are many surveys
asking happiness question, for instance, World Values Surveys, General Social
Surveys and Eurobarometers, but they do not survey energy use. 

% Then there are needs v wants--how much energy we really need v what we want
% (which is endless). For instance, we need temperature in a dwelling to be above
% a freezing point, but we want temperature to be in 80s in winter, while 60s
% would be fine. We need some transportation--train or perhaps some car, but we
% want an SUV, and so forth. The point is that in the US, easily half of energy
% consumption is a want (for instance Texans consume twice as much as New
% Jersians--they would be fine if they consume at levels of New Jersey).% stover14

%  And equity and environmental justice--it is wildly
% recognized that there is income inequality \cite{piketty14} but there is
%  also plenty of environmental injustice--for instance Americans consume much
%  more resources than Indians, and of course within the US, recource consumption
%  varies enormously as well--for insance Texas consumes twice as much energy per
%  capita than New Jersey TODOciteFromOtehrMaterialHere. It is plainly no fair to
%  require poor countries such as India, or poor areas within countries or poor
%  people to care as much about conservation as the rich.
%  Furthermore, inequality in energy
%  consumption is in some ways more important than inequality in income, because
%  income is unlimited, while energy is (still most of energy comes from fossil
%  fuels TODOdbleCheck).  % stover14

% Energy efficiency is not ultimate solution and actually often backfires--it is
% like building more reoads or more lanes for traffic or loosening your belt for
% big belly--traffic increases, belly grows further and energy consumption grows
% furtehr too!
% %http://www.nytimes.com/2014/10/09/opinion/the-problem-with-energy-efficiency.html?hp&action=click&pgtype=Homepage&module=c-column-top-span-region&region=c-column-top-span-region&WT.nav=c-column-top-span-region&_r=0
% %hwys and loosening belt come from duany book

% Energy use and environmental iussies are  increasingy political in the US, and
% may even become a key issue in the near furtre. There is now a surfge in
% political advertising \cite{davenport_nyt_oct21_14}.


\section*{\large \bf Materials and Methods}
%boilerplate following
% http://www.pnas.org/content/109/49/19949.full.pdf  
% http://www.pnas.org/content/112/3/725.full.pdf
% http://www.pnas.org/content/109/25/9775.full.pdf
All happiness measures come from surveys representative of given areas. Such measures are reasonably valid and
reliable \cite{diener13b}. % , consistent and stable? or means same as valid, reliable?
One caveat is cross-cultural comparability \cite{diener03b}. The cross
country results that we report, however, show  strong relationship and it is very
unlikely that the whole effect is due to measurement error. Furthermore, we mostly use data
within the US. % LATER Energy measures  discuss their reliability, validity etc just google around.
We focus on simple relationships. Multivariate regressions are
discussed in the supplementary online material. 


\noindent\textbf{ACKNOWLEDGMENTS.} We thank... 


\newpage
%\bibliography{/home/aok/papers/root/tex/ebib.bib}


\begin{thebibliography}{10}

\bibitem{mackay08}
MacKay D (2008) {\em Sustainable Energy-without the hot air}.
\newblock (UIT Cambridge).

\bibitem{arrow04}
Arrow K et~al. (2004) Are we consuming too much?
\newblock {\em Journal of Economic Perspectives} pp. 147--172.

\bibitem{soytas07}
Soytas U, Sari R, Ewing BT (2007) Energy consumption, income, and carbon
  emissions in the united states.
\newblock {\em Ecological Economics} 62(3):482--489.

\bibitem{stiglitz09al}
Stiglitz J, Sen A, Fitoussi J (2009) Report by the commission on the
  measurement of economic performance and social progress.
\newblock {\em Available at www.stiglitz-sen-fitoussi.fr}.

\bibitem{kenny_businessweek_aug_29_14}
Kenny C (2014) Poor countries shouldn't sacrifice growth to fight climate
  change.
\newblock {\em Businessweek} (August).

\bibitem{gordon_wsj_may_29_14}
Gordon K (2014) Cutting energy without self sacrifice.
\newblock {\em Wall Street Journal} (May).

\bibitem{carter_pbs_apr_18_77}
Carter J (1977) Proposed energy policy.
\newblock {\em Public Broadcasting Service} (May).

\bibitem{smil05}
Smil V (2005) Creating the twentieth century: technical innovations of
  1867-1914 and their lasting impact.
\newblock {\em OUP Catalogue}.

\bibitem{mackerron09}
MacKerron G, Mourato S (2009) Life satisfaction and air quality in london.
\newblock {\em Ecological Economics} 68(5):1441--1453.

\bibitem{gandelman12}
Gandelman N, Piani G, Ferre Z (2012) Neighborhood determinants of quality of
  life.
\newblock {\em Journal of Happiness Studies} 13:547--563.

\bibitem{ferreira13}
Ferreira S et~al. (2013) Life satisfaction and air quality in europe.
\newblock {\em Ecological Economics} 88:1--10.

\bibitem{kallis11}
Kallis G (2011) In defence of degrowth.
\newblock {\em Ecological Economics} 70(5):873--880.

\bibitem{kallis12}
Kallis G, Kerschner C, Martinez-Alier J (2012) The economics of degrowth.
\newblock {\em Ecological Economics} 84:172--180.

\bibitem{pretty13}
Pretty J (2013) The consumption of a finite planet: well-being, convergence,
  divergence and the nascent green economy.
\newblock {\em Environmental and Resource Economics} 55(4):475--499.

\bibitem{mazur11}
Mazur A (2011) Does increasing energy or electricity consumption improve
  quality of life in industrial nations?
\newblock {\em Energy Policy} 39(5):2568--2572.

\bibitem{mazur74}
Mazur A, Rosa E (1974) Energy and life-style.
\newblock {\em Science (New York, NY)} 186(4164):607.

\bibitem{easterlin74}
Easterlin RA (1974) {\em Does Economic Growth Improve the Human Lot?} eds.{}
  David PA, Reder MW.
\newblock (New York: Academic Press, Inc.), Vol.{}~89, pp. 98--125.

\bibitem{easterlin12}
Easterlin RA, Morgan R, Switek M, Wang F (2012) China's life satisfaction,
  1990--2010.
\newblock {\em Proceedings of the National Academy of Sciences}
  109(25):9775--9780.

\bibitem{bea-2-8-5}
BEA (2014) Table 2.8.5. personal consumption expenditures by major type of
  product, monthly.
\newblock {\em Bureau of Economic Analysis}.

\bibitem{kahneman97ws}
Kahneman D, Wakker PP, Sarin R (1997) Back to bentham? {E}xplorations of
  experienced utility.
\newblock {\em The Quarterly Journal of Economics} 112(2):375--405.

\bibitem{brickman78cj}
Brickman P, Coates D, Janoff-Buman R (1978) Lottery winners and accident
  victims: Is happiness relative?
\newblock {\em Journal of Personality and Social Psychology} 36:917--927.

\bibitem{veblen05a}
Veblen T (2005) {\em Conspicuous consumption}.
\newblock (ePenguin) Vol.{}~38.

\bibitem{veblen05b}
Veblen T (2005) {\em The theory of the leisure class; an economic study of
  institutions}.
\newblock (Aakar Books).

\bibitem{csikszentmihalyi99}
Csikszentmihalyi M (1999) If we are so rich, why aren't we happy?
\newblock {\em American psychologist} 54(10):821.

\bibitem{frank04}
Frank RH (2004) How not to buy happiness.
\newblock {\em Daedalus} 133(2):69--79.

\bibitem{frank05}
Frank RH (2005) {\em Does Absolute Income Matter} eds.{} Bruni L, Porta PL.
\newblock (Oxford University Press).

\bibitem{frank12}
Frank R (2012) {\em The Darwin economy: Liberty, competition, and the common
  good}.
\newblock (Princeton University Press).

\bibitem{kahneman10}
Kahneman D, Deaton A (2010) High income improves evaluation of life but not
  emotional well-being.
\newblock {\em Proceedings of the National Academy of Sciences}
  107(38):16489--16493.

\bibitem{winfrey13}
Winfrey EMV (2013) Ph.D. thesis (Georgetown University).

\bibitem{diener13b}
Diener E, Inglehart R, Tay L (2013) Theory and validity of life satisfaction
  scales.
\newblock {\em Social Indicators Research} 112(3):497--527.

\bibitem{diener03b}
Diener E, Suh EM, eds. (2003) {\em Culture and Subjective Well-Being}.
\newblock (MIT Press).

\end{thebibliography}



\newpage
\section*{\huge ONLINE SUPPLEMENATRY MATERIAL}

\tableofcontents

\section{Country-level additional information}

 \begin{scriptsize}  \begin{center} \begin{longtable}{llllllllllllll} \caption{
Key
variables
for
each
country.
Sorted
on
happiness.
Note:
if
country
was
observed
in
more
than
one
year,
values
are
averaged"}
\label{ls}
\\
\hline
\multicolumn{1}{p{.75in}}{"Country
Name"}
&
\multicolumn{1}{p{.75in}}{"ISO
3166
alpha-3
code"}
&
\multicolumn{1}{p{.75in}}{"happiness"}&
\multicolumn{1}{p{.75in}}{"energy
use,
pc"}&
\multicolumn{1}{p{.75in}}{"PCGDP"}&
\multicolumn{1}{p{.75in}}{"co2
emissions,
pc"}&
\multicolumn{1}{p{.75in}}{"female
life
expectancy"}
&
\multicolumn{1}{p{.75in}}{""}
\\
\hline
\endfirsthead
\multicolumn{3}{p{.75in}}
{{\bfseries
\tablename\
\thetable{}
--
continued
from
previous
page}}
\\
\hline
\multicolumn{1}{p{.75in}}{"Country
Name"}
&
\multicolumn{1}{p{.75in}}{"ISO
3166
alpha-3
code"}
&
\multicolumn{1}{p{.75in}}{"happiness"}&
\multicolumn{1}{p{.75in}}{"energy
use,
pc"}&
\multicolumn{1}{p{.75in}}{"PCGDP"}
&
\multicolumn{1}{p{.75in}}{"co2
emissions,
pc"}&
\multicolumn{1}{p{.75in}}{"female
life
expectancy"}
&
\multicolumn{1}{p{.75in}}{""}\\
\hline
\endhead
\hline
\multicolumn{5}{r}{{Continued
on
next
page}}
\\
\endfoot
\hline
\endlastfoot

Tanzania&TZA&3.9&407&315&0.1&51\\
Zimbabwe&ZWE&3.9&770&681&1.0&43\\
Armenia&ARM&4.3&593&748&1.0&73\\
Moldavia&MDA&4.6&973&720&1.9&71\\
Georgia&GEO&4.8&748&1,328&1.2&76\\
Iraq&IRQ&4.8&887&1,877&3.9&72\\
Pakistan&PAK&4.9&438&588&0.7&64\\
Ukraine&UKR&4.9&2,936&1,567&7.5&74\\
Latvia&LVA&4.9&1,840&3,470&3.9&75\\
Belarus&BLR&4.9&3,483&1,948&5.9&75\\
Bulgaria&BGR&4.9&2,580&3,176&6.3&75\\
Albania&ALB&5.0&532&2,034&0.9&77\\
Rwanda&RWA&5.0&&304&0.1&60\\
Lithuania&LTU&5.0&2,490&4,561&4.3&77\\
Ethiopia&ETH&5.0&380&187&0.1&60\\
Estonia&EST&5.0&4,043&5,325&12.7&75\\
Romania&ROU&5.3&1,819&3,970&4.5&75\\
Russia&RUS&5.3&4,978&5,006&11.5&73\\
Azerbaijan&AZE&5.4&1,444&684&3.8&69\\
Macedonia&MKD&5.4&1,341&2,541&6.0&76\\
Egypt&EGY&5.6&785&1,298&2.4&72\\
Burkina Faso&BFA&5.6&&424&0.1&54\\
Bosnia-Herzegovinia&BIH&5.6&1,001&2,177&5.3&77\\
Uganda&UGA&5.7&&273&0.1&49\\
Morocco&MAR&5.7&423&1,913&1.5&71\\
Algeria&DZA&5.7&883&2,676&2.8&71\\
India&IND&6.0&425&567&1.1&62\\
Zambia&ZMB&6.1&611&669&0.1&51\\
Bangladesh&BGD&6.1&141&339&0.2&65\\
Mali&MLI&6.1&&472&0.0&52\\
Slovakia&SVK&6.1&3,641&8,882&7.7&76\\
Ghana&GHA&6.1&402&540&0.4&61\\
South Korea&KOR&6.2&3,024&12,134&7.5&77\\
Croatia&HRV&6.2&1,610&7,188&4.1&77\\
Czech Republic&CZE&6.4&4,424&9,769&11.9&77\\
Hungary&HUN&6.4&2,566&7,942&7.0&74\\
Jordan&JOR&6.4&1,127&2,289&3.5&74\\
Iran&IRN&6.4&2,260&2,636&6.6&72\\
Hong Kong&HKG&6.4&1,859&26,650&6.0&84\\
Turkey&TUR&6.5&1,160&6,051&3.2&73\\
Kyrgyzstan&KGZ&6.5&512&457&1.1&72\\
Peru&PER&6.6&479&2,605&1.2&74\\
South Africa&ZAF&6.6&2,647&5,030&9.1&60\\
Japan&JPN&6.6&3,708&31,106&9.1&83\\
Nigeria&NGA&6.7&730&559&0.5&47\\
Poland&POL&6.7&2,774&6,934&9.4&77\\
Philippines&PHL&6.7&485&1,048&0.9&70\\
Vietnam&VNM&6.8&450&649&1.0&79\\
Malaysia&MYS&6.8&2,420&5,756&6.5&76\\
Slovenia&SVN&6.9&3,347&15,003&7.6&80\\
China&CHN&6.9&1,030&1,163&3.2&73\\
France&FRA&6.9&4,194&34,413&6.0&84\\
Italy&ITA&6.9&3,172&30,814&8.2&84\\
Germany&DEU&6.9&4,168&32,716&10.2&81\\
Indonesia&IDN&6.9&778&1,217&1.5&70\\
Spain&ESP&7.0&2,771&22,412&6.8&82\\
Israel&ISR&7.0&2,973&18,735&10.2&81\\
Venezuela&VEN&7.1&2,382&5,362&5.8&75\\
Dominican Republic&DOM&7.1&684&2,700&2.2&73\\
Andorra&AND&7.1&&31,269&7.1&\\
Argentina&ARG&7.2&1,593&5,217&3.8&76\\
Chile&CHL&7.2&1,453&6,085&3.4&79\\
Thailand&THA&7.2&1,587&2,946&4.0&76\\
Singapore&SGP&7.2&5,030&24,811&11.3&81\\
Saudi Arabia&SAU&7.3&5,312&12,337&14.3&75\\
Uruguay&URY&7.3&914&5,142&1.8&79\\
Cyprus&CYP&7.3&2,202&22,910&7.4&81\\
Brazil&BRA&7.4&1,064&4,424&1.7&73\\
El Salvador&SLV&7.5&640&2,513&1.0&74\\
USA&USA&7.5&7,795&39,989&19.5&80\\
United Kingdom&GBR&7.6&3,693&35,563&9.1&81\\
Australia&AUS&7.6&5,127&27,234&16.7&81\\
Sweden&SWE&7.7&5,672&36,336&5.9&82\\
Netherlands&NLD&7.7&4,700&40,385&10.2&82\\
New Zealand&NZL&7.8&4,205&24,918&8.2&81\\
Canada&CAN&7.8&8,205&35,027&17.1&82\\
Norway&NOR&7.8&5,719&60,805&9.1&82\\
Finland&FIN&7.8&5,845&28,897&11.0&80\\
Mexico&MEX&7.9&1,428&7,186&3.7&75\\
Guatemala&GTM&8.0&620&2,146&1.0&73\\
Switzerland&CHE&8.1&3,408&49,370&5.5&82\\
Puerto Rica&PRI&8.3&&18,806&&80\\
Colombia&COL&8.3&699&3,263&1.6&75\\
\end{longtable} \end{center} \end{scriptsize}   


\subsection{Country-level additional regressions: taking into account more factors}


Table \ref{var_des} lists variables used and their definitions for cross-country
analysis. 

\begin{table}[H]\centering\footnotesize
 \caption{\label{var_des} Variable definitions}
\begin{tabular} {p{1.5in}p{4.5in}}   \hline
name & description   \\ \hline
  happiness & "All things considered, how satisfied are you with your life as a whole these days?" 1="dissatisfied" to 10="satisfied"; WVS \\
  PCGDP & GDP per capita (constant 2005 US\$); Code: NY.GDP.PCAP.KD; "GDP per capita is gross domestic product divided by midyear population. GDP is the sum of gross value added by all resident producers in the economy plus any product taxes and minus any subsidies not included in the value of the products. It is calculated without making deductions for depreciation of fabricated assets or for depletion and degradation of natural resources. Data are in constant 2005 U.S. dollars."; WB \\
  energy use, pc & Energy use (kg of oil equivalent per capita); Code: EG.USE.PCAP.KG.OE "Energy use refers to use of primary energy before transformation to other end-use fuels, which is equal to indigenous production plus imports and stock changes, minus exports and fuels supplied to ships and aircraft engaged in international transport."; WB \\
  unemployment, \% & Unemployment, total (\% of total labor force) (modeled ILO estimate); Code: SL.UEM.TOTL.ZS; "Unemployment refers to the share of the labor force that is without work but available for and seeking employment."; WB \\
  co2 emissions, pc & CO2 emissions (metric tons per capita); Code: EN.ATM.CO2E.PC; "Carbon dioxide emissions are those stemming from the burning of fossil fuels and the manufacture of cement. They include carbon dioxide produced during consumption of solid, liquid, and gas fuels and gas flaring."; WB \\
  female life expectancy & Life expectancy at birth, female (years); Code: SP.DYN.LE00.FE.IN; "Life expectancy at birth indicates the number of years a newborn infant would live if prevailing patterns of mortality at the time of its birth were to stay the same throughout its life." WB\\
  road sector gasoline fuel consumption, pc & Road sector gasoline fuel consumption per capita (kg of oil equivalent); Code: IS.ROD.SGAS.PC; "Gasoline is light hydrocarbon oil use in internal combustion engine such as motor vehicles, excluding aircraft."; WB \\
  percent urban & population (\% of total); Code: SP.URB.TOTL.IN.ZS; "Urban population refers to people living in urban areas as defined by national statistical offices. It is calculated using World Bank population estimates and urban ratios from the United Nations World Urbanization Prospects." WB\\
  maximum temperature in January & "near-surface temperature maximum (degrees Celsius)" ; TYN\_CY \\
  maximum temperature in July & "near-surface temperature maximum (degrees Celsius)" ; TYN\_CY \\
\hline\end{tabular}\end{table}


% WDH:
% \url{http://www1.eur.nl/fsw/happiness/hap_nat/nat_fp.php?mode=8}

{\scriptsize \noindent Variable sources. WVS: World Values Survey \url{www.worldvaluessurvey.org};
TYN\_CY: Tyndall Centre (\url{www.tyndall.ac.uk}) Mitchell,T.D., Hulme,M., and
New,M., 2002: Climate data for political areas. Area
34:109-112. \url{http://www.cru.uea.ac.uk/~timm/cty/obs/TYN_CY_1_1_var-table.html};
WB: World Bank \url{http://data.worldbank.org}}

\begin{figure}[H]
 \includegraphics[width=6in]{graphsAndTables/couWdhEneGdpHly.pdf}\centering
\caption{Happy Life Years as a Dependent Variable}\label{hly}
\end{figure}

It is critical to control for percent urban or population density, because
urban or dense areas are less happy and more energy efficient, and hence
omission of this variable leads to positive bias on energy use--the estimated
coefficient   is larger than it should be in a bivariate case. Indeed, bivariate
relationship appears positive. 

The bottom line is that there appears to be weak relationship between electricity
consumption and happiness, but it disappears or indeed becomes negative when
taking into account social support. One explanation is that people who consume
more electricity, need to work more, and have less time for social interaction, which
is important for happiness. Per Robert Putnam's Bowling Alone, we have less and
less social interaction and per Robert Frank--we tend to  focus more on non-pecuniary domain!

Regression results are set in table \ref{regA}. First we start with an ols
model. All ols regressions control for year dummies--again, different countries
were surveyed in different years. Energy consumption results in greater
happiness (ols1), but when controlling for PCGDP, the relationship disappears
(ols2). Addition of percent urban, unemployment rate and life expectancy (ols3)
makes the relationship actually significantly negative.  Addition of maximum
temperatures in January and July (ols4) makes it positive again but still
insignificant.  A very interesting thing occurs in column ols5--addition of
$CO_2$ makes energy positive and significant.\footnote{The two variables are
  correlated at .91. And $CO_2$ is negative as expected. Energy correlates at .38
  with happiness, and $CO_2$ correlates at .29 (positive, too).}
 Hence, energy consumption would increase happiness if not emissions. This may
 point to clean energy (wind, solar, etc) as a solution. 

Furthermore, in such a diverse sample of countries, there is some unobserved
heterogeneity. Two fixed effects model follow (temperature drops out, because we
 only have temperature for country for one time period). Yet comfortingly
 results are similar to ols, when not controlling for $CO_2$, results are weakly
 positive (fe1), but when adding $CO_2$ control results become significant
 (fe2). Again, we interpret it  that energy consumption WITHOUT pollution would
 contribute more to happiness than just energy consumption.  

\begin{figure}[H]
 \includegraphics[width=6in]{graphsAndTables/ols4ols5.pdf}\centering
\caption{WVS all countries, first is ols4; second is ols5--the latter controls
  for co2--if we could have energy without pollution it would increase happiness! }\label{ols4ols5}
\end{figure}
{\scriptsize Note: Country codes are in table \ref{ls} in supplementary
  material. If country was observed in more than one year, the data were averaged.}


\begin{table}[H]\centering \caption{regA} \label{regA} \begin{scriptsize} \begin{tabular}{p{1.4in}p{.43in}p{.43in}p{.43in}p{.43in}p{.43in}p{.43in}p{.43in}p{.43in}p{.43in}p{.43 in}p{.43in}p{.43 in}}\hline                     &        ols1   &        ols2   &        ols3   &        ols4   &        ols5   &         fe1   &         fe2   \\
energy use, pc      &       0.000***&      -0.000+  &      -0.000*  &       0.000   &       0.000** &       0.001+  &       0.001*  \\
PCGDP               &               &       0.000***&       0.000***&       0.000** &       0.000*  &       0.000   &      -0.000   \\
percent urban       &               &               &       0.018** &      -0.001   &      -0.001   &       0.019   &       0.030   \\
unemployment, \%     &               &               &      -0.026*  &      -0.008   &      -0.002   &       0.010   &       0.009   \\
female life expectancy&               &               &       0.006   &       0.050** &       0.057***&      -0.002   &      -0.003   \\
maximum temperature in January&               &               &               &       0.045***&       0.049***&               &               \\
maximum temperature in July&               &               &               &      -0.019*  &      -0.014   &               &               \\
co2 emissions, pc   &               &               &               &               &      -0.108** &               &      -0.262   \\
constant            &       6.592***&       6.693***&       5.473***&       2.725*  &       1.921   &       3.657   &       3.245   \\
N                   &         163   &         162   &         139   &         139   &         139   &         139   &         139   \\
 \hline\multicolumn{6}{l}{+p$<$0.10 *p$<$0.05 **p$<$0.01 ***p$<$0.001; robust standard errors} \end{tabular}\end{scriptsize}\end{table}

\begin{verbatim}
             |   ene   gdp   urb   un   lexp  janMax julMax
-------------+---------------------------------------------------------------
         ene |   1.0 
         gdp |   0.7*  1.0 
         urb |   0.5*  0.5*  1.0 
          un |  -0.1* -0.2*  0.0   1.0 
        lexp |   0.5*  0.5*  0.7* -0.0   1.0 
      janMax |  -0.4* -0.3* -0.1  -0.0  -0.4*  1.0 
      julMax |  -0.3* -0.3* -0.3*  0.0  -0.2*  0.2*  1.0 
         co2 |   0.9*  0.6*  0.5* -0.0   0.4* -0.4* -0.2*

\end{verbatim}

\section{State-level additional results}
\subsection{How do we use energy in the US?}

Energy use is slightly increasing in the US
\url{http://www.eia.gov/todayinenergy/detail.cfm?id=4690}, and quite a bit in
the World. 
Energy use in the US is fairly flat over past 40 years at 70m btu
pc.\url{http://www.eia.gov/todayinenergy/detail.cfm?id=3590}, and coasts consume
less than inland middle
\url{http://energy.gov/maps/2009-energy-consumption-person?page=0%2C1}. 
Use by sector in the US: 22\% residential, 18\% commercial, 32\% industrial, and
28\% transportation.\url{http://www.eia.gov/consumption/}. 


For the US CBO produced a handy chart \url{http://www.cbo.gov/publication/43232} or \url{http://www.cbo.gov/sites/default/files/43232-infographic-EnergySecurity.pdf}

Some popular media description
\url{http://www.theatlantic.com/technology/archive/2013/08/a-very-short-history-of-how-americans-use-energy-at-home/278329/}

We know how energy is used in residential sector in the US ( we also know it for
other sectors but we do not focus on them)--all data are here
\url{http://www.eia.gov/forecasts/aeo/MT_residentialdemand.cfm}. For residential
sector
\url{http://www.eia.gov/oiaf/aeo/tablebrowser/aeo_query_server/?event=ehExcel.getFile&study=AEO2014&region=0-0&cases=ref2014-d102413a&table=4-AEO2014&yearFilter=0}:
table AEO2014: 


%LATER may see footnotes for this table
\begin{table}[H]\centering\footnotesize
\caption{\label{freq_im_god} Total Energy Consumption by End Use; quadrillion
  Btu, 2011}
\begin{tabular}{lll}   \hline 
Space Heating&	5.6\\
Space Cooling&	2.6\\
Water Heating&	2.7\\
Refrigeration&	1.2\\
Cooking&	0.6\\
Clothes Dryers&	0.7\\
Freezers&	0.2\\
Lighting&	2\\
Clothes Washers&	0.1\\
Dishwashers 1/	0.307437
Televisions and Related Equipment&	1\\
Computers and Related Equipment &	0.4\\
Furnace Fans and Boiler Circulation Pumps&	0.4\\
Other Uses&	3.7\\\hline
\end{tabular}\end{table}

How is electricity used in United States homes? This is an important
consideration because it really shows what we do with this electricity--how we
consume it, what are the end uses. Data are shown in table
\ref{eleEndUse}. Furthermore end uses of energy changed over time, for instance
 from 1993 to 2009: applianes share increased from 24\% to 35\% and space
 heating dropped from 53\% to 41\%
 \url{http://www.eia.gov/todayinenergy/detail.cfm?id=10271&src=%E2%80%B9%20Consumption%20%20%20%20%20%20Residential%20Energy%20Consumption%20Survey%20%28RECS%29-b1}. 
Also, the good news is that average energy consumption per household dropped from 114 m BTU in 1980 to
90 m BTU in 2009 \url{http://www.eia.gov/consumption/residential/reports/2009/consumption-down.cfm?src=%E2%80%B9%20Consumption%20%20%20%20%20%20Residential%20Energy%20Consumption%20Survey%20%28RECS%29-b5}.  


\begin{table}[H]\centering\footnotesize
\caption{\label{eleEndUse}  Estimated U.S. Residential \underline{Electricity} Consumption by End
  Use, 2012 \url{www.eia.gov/tools/faqs/faq.cfm?id=96&t=3}}
\begin{tabular} {llll}   \hline 
End Use&Quadrillion Btu &Billion kilowatthours& \% Share of total\\\hline 
Space cooling&0.85&250&18.00\%\\
Lighting&0.64&186&14.00\%\\
Water heating&0.45&130&9.00\%\\
Refrigeration&0.38&111&8.00\%\\
Televisions and related equipment 1&0.33&98&7.00\%\\
Space heating&0.29&84&6.00\%\\
Clothes dryers&0.2&59&4.00\%\\
Computers and related equipment2&0.12&37&3.00\%\\
Cooking&0.11&31&2.00\%\\
Dishwashers3 &0.1&29&2.00\%\\
Furnace fans and boiler circulation pumps&0.09&28&2.00\%\\
Freezers&0.08&24&2.00\%\\
Clothes washers3&0.03&9&1.00\%\\
Other uses4&1.02&299&22.00\%\\
Total consumption&4.69&1375&\\\hline
\end{tabular}\end{table}

there is also consumption by end use by census region or climate region

\url{http://www.eia.gov/consumption/residential/data/2009/index.cfm?view=consumption}

\section{Over-time movement}

\begin{figure}[H]
 \includegraphics[width=6in]{graphsAndTables/cenDivLsYr.pdf}\centering
\caption{Happiness and energy use across census divisions over time.}\label{cenDivLsYr}
 \end{figure}

\end{spacing}
\end{document}
